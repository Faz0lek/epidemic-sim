\documentclass[a4paper,11pt]{article}

\usepackage[left=2cm, top=3cm, text={17cm, 24cm}]{geometry}
\usepackage[czech]{babel}
\usepackage[utf8]{inputenc}
\usepackage{times}
\usepackage[unicode]{hyperref}
\hypersetup{colorlinks = true, hypertexnames = false}

\begin{document}
	\begin{titlepage}
		\begin{center}
			\textsc{\Huge Vysoké učení technické v~Brně\\
				\vspace{0.4em}\huge Fakulta informačních technologií}
			
			\vspace{\stretch{0.382}}
			
			{\LARGE Typografie a~publikování\,--\,4. projekt\\
				\Huge Bibliografické citace\\ \vspace{0.3em}}
			
			\vspace{\stretch{0.618}}
			
			{\Large \today \hfill Martin Kostelník}
		\end{center}
	\end{titlepage}

	\section{Typografie}
		Typografie je obor zabývající se tiskovým písmem. Skládá se ze dvou částí, mikrotypografie a makrotypografie. Mikrotypografie se zabývá zvyšováním estetické kvality sazby. Makrotypografie naopak umístěním písma na stránku. Více o jednotlivých částech se dá najít v \cite{macromicro}.
		
		\subsection{Historie}
			Dějiny typografie začínají v polovině 15.\,století. Dále si zásluhy za rozvoj typografie bere Johannes Gutenberg, který v roce 1439 vynalezl sázení pomocí pohyblivých komponent. \cite{gutten}
			
			\vspace{10pt}
			
			Zásluhy za typografii si bere také Čech Vojtěch Preissig, který zkoumal vzájemné vztahy barev a tvarových konstrukcí. Více o jeho životě a tvorbě v \cite{Hoskova2013}.
			
		\subsection{Znakové sady}
			Písmo má za sebou dlouhý vývoj. Dnešní písma se odvozují od písem z Egypta a Mezopotámie. Písmo se vývijelo mnoho století. Více informací o písmech lze najít v \cite{Cerny1999}.
			
			O fontech v sázecím systému \LaTeX se můžete dočíst v \cite{olsak}.
			
	\section{\LaTeX}
		\LaTeX je sázecí systém k vytváření složitých matematických dokumentů. Práce v \LaTeX u však není úplně jednoduchá. Nicméně na internetu existuje velké množštví návodů. Začátečník by měl začít tvorbou jednoduchých dokumentů, viz. \cite{overleaf}.
		
		\subsection{Programy}
			Existuje více programů, ve kterých lze dokumenty sázet. Některé ukazují i pdf náhled, některé napovídají syntax. Porovnání různých programů se věnuje článek \cite{programs}
		
		\subsection{Matematické výrazy}
			\LaTeX slouží primárně k sázení matematických textů. Pro sazbu se používá matematické prostředí uvozené \$\$. \LaTeX dále podporuje sazbu rovnic, výrazů, tabulek i matic. Matematický režim je více popsán v \cite{mathlatex}.
		
			\vspace{10pt}
			
			Použití \LaTeX u je také velice výhodné pokud chceme sázet matematické výrazy v prostředí webů. Bez použití \LaTeX u je to velice obtížné, protože základní protokoly nemají žádnou podporu pro matematické výrazy. Jak správně psát takovéto výrazy ve webovém prostředí můžete najít v \cite{webmath}. 
			
		\subsection{Citace}
			Pro vytvoření citací musíme nejprve vytvořit seznam použité literatury a následně vložit odkazy na ně do samotného textu. Pro práci s citacemi existuje více prostředí. například \emph{thebibliography} nebo BiTeX. Podrobněji jsou citace v \LaTeX u popsány na \cite{citation}.

	\newpage
	\bibliographystyle{czechiso}
	\renewcommand{\refname}{Literatura}
	\bibliography{proj4}
\end{document}