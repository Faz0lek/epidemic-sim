\documentclass[a4paper,11pt]{article}

\usepackage[left=2cm, top=3cm, text={17cm, 24cm}]{geometry}
\usepackage[czech]{babel}
\usepackage[utf8]{inputenc}
\usepackage{times}
\usepackage[unicode]{hyperref}
\hypersetup{colorlinks = true, hypertexnames = false}

\begin{document}
	\begin{titlepage}
		\begin{center}
			\textsc{\Huge Vysoké učení technické v~Brně\\
				\vspace{0.4em}\huge Fakulta informačních technologií}
			
			\vspace{\stretch{0.382}}
			
			{\LARGE Modelování a simulace\\
				\Huge Predikce epidemie COVID-19\\ \vspace{0.3em}}
			
			\vspace{\stretch{0.618}}
			
			{\Large \hfill Radek Švec (xsvecr)\\ \today \hfill Martin Kostelník (xkoste12)}
		\end{center}
	\end{titlepage}

	\section{Úvod}
		Tématem práce jsou epidemiologické modely na makroúrovni, konkrétně předpovídání šíření pandemie COVID-19 na území Itálie a České republiky. Tato práce implementuje již existující model vytvořený Giuliou Giordanem a dalšími \cite{source}, který předpovídá vývoj pandemie na území Itálie. Smyslem projektu bude předpovědět šíření nemoci COVID-19 na území České republiky v následujících měsících, vzít v úvahu různá opatření proti šíření a zhodnotit, zda byly dosavadně zavedené opatření dostačující, případně zda bylo rozvolnění těcho opatření správným rozhodnutím.
		
	\subsection{Autoři a zdroje informací}
		Projekt byl vypracován
			
	\section{Popis abstraktího modelu}
	\section{Implementace simulačního modelu}
	\section{Zopakování experimentů}
	\section{Aplikování modelu na Českou republiku}
	\subsection{Experimenty}
			
	\section{Závěr}
		asdfasdfasdfasdfasdfasdfasdfasdfasdfasdfasdfasdfasdf
		asdfasdfasdfasdfasdfasdfasdfasdf

	\newpage
	\bibliographystyle{czechiso}
	\renewcommand{\refname}{Literatura}
	\bibliography{doc}
\end{document}